\documentclass[../memory.tex]{subfiles}

\begin{document}
\chapter{Future work}
As a final chapter of this project, the goal is to point out different points
that could: extend the features of the application, improve current features,
and improve the current developer experience and the application environment. It
is important to note that this chapter will only focus in the future work that
could be done regarding the frontend application, which will most likely have
implications in the backend.
\\[8pt]
Firstly, regarding future features, the application could have a subscrition
system, that would limit the user in the amount of travels or queries they can
do, for a certain period of time. These limitations would have to be
implemented in the frontend, ensuring that the user can or cannot do certain
actions, regarding their status.
\\
Another great addition would be the possibility for the user to know how many
alerts have been sent, as it was first designed\footnote{In the dashboard
	design, each item of the table has a red dot with a text of how many new alerts
	there are}. However, due to the lack of time and not being part of the MVP, this
feature has not been implemented in the backend.
\\[8pt]
Secondly, regarding current features to be improved, I would strongly recommend
spending time in tracking user behaviour, in order to see how the application it
is used. This tracking is normally avoided by start-ups or small companies,
because the harsh it can be to implement. Nonetheless, it can provide rich
insights about the users using the application, and how it is used.
\\
Furthermore, even though the user experience is decent (performant, concise and
intuitive UI), I strongly believe it can be improved. Small animations, improved
page transitions or even a more colourful UI, would provide a nearly excellent
UI and UX.
\\[8pt]
Thirdly, regarding current features to be improved, there are a few that are
better listed than explained:
\begin{enumerate}
	\item\textbf{Improved code splitting}. Even though this is already done by
	default in Next.js, each page still serves a generous amount of bundled
	code. The frontend should be as light as possible, in order to reduce the JS
	sent to the user. Since the architecture of the projet has been strictly
	layered, a good performance improvement could be an improved code-splitting
	system.
	\item\textbf{Taking full advantage of Next.js and Vercel}. During the
	implementation of this project, Vercel has launched many new features that
	provide many enhancements to the latests Next.js versions. Even though this
	project uses Next.js 13, most of these improvements work for versions 13.x.
	Furthermore, since the deployment is done in Vercel, even more enhancements
	can be used.
	\item\textbf{E2e testing implementation}. Due to the lack of time, and the
	work that is required to configure e2e tests, this have not been done.
	Thus, spending time in implementing e2e tests, with an environment as close
	to production as possible, would allow the developers to ship new features
	ensuring that the shipped code does not break current functionalities.
	\item\textbf{Improved CI pipeline}. The current pipeline is decent, yet
	not fast enough. Currently, for each run, all npm packages are installed,
	which leads to an execution time of around 5 minutes, 4 of which are just
	because of the installation. This issue can be addressed by many solutions
	such as GitHub actions caching, or even Nx Cloud. The idea of this feature
	would be to implement such caching in order to provide a better developer
	experience, as well as reducing the resources used in pipelines.
\end{enumerate}
\end{document}
