\documentclass[../memory.tex]{subfiles}

\begin{document}
\chapter{Cost analysis}
After conducting a thorough planning and being conscious about all the tasks
that are required to achieve the expected result, this chapter contains an
analysis of the costs of this project, detailing the different investments that
could be made.
\\[8pt]
It has been decided to divide investments dedicated to the project into two main
categories: human related resources and non-human related resources.
\\[8pt]
Finally, provisions for contingencies and any unforeseen expenses that may arise
are also taken into account.
\section{Human costs}
In any project, one of the key factors and usually the most expensive is the
presonnel that will bring such project to live. In this case, the project is
being developed by a group of students, as part of their degree's curriculum,
meaning that there is no direct cost linked to the work done.
\\[8pt]
In order to analyse the human costs for this project, it is important to
consider the different actors involved in it, and the time that will be spent in
the project.
\begin{enumerate}
	\item\textbf{Students}. Students will require the major part of the budget. As
	established by the university, the final project equals to 10 ECTS credits,
	which corresponds to a time expenditure between 250 and 300 hours, each
	student. In order to simplify the calculation, it has been supposed that
	each student will spend 275 hours.
	\item\textbf{Tutor}. On the other side, during the development of the project,
	there were a set of scheduled meetings that involved the students and the
	tutor of the project. The project lasts around 22 weeks, and it has been
	estimated that, including the work outside the meetings plus the meetings,
	the tutor would spend around 3h each week.
\end{enumerate}
\vspace*{8pt}
\begin{tabularx}{\textwidth}{|X|X|X|X|}
	\hline
	\rowcolor{rowColor}
	Actor                 & Hours/actor         & Num. of actors & Total hours \\
	\hline
	Student               & 275 hours           & 4              & 1100 hours  \\
	\hline
	Tutor                 & 66 hours            & 1              & 66 hours    \\
	\hline
	\multicolumn{3}{X|}{} & \textbf{1166 hours}                                \\
	\cline{4-3}
\end{tabularx}
\captionof{table}{Human resources costs}
\vspace*{8pt}
Given this information, the total cost of each actor can be calculated through:
\begin{center}
	\[ T_c = C_{ha}(\texteuro/h) * N_h \]
\end{center}
In this equation, $T_c$ stands for the total actor cost; $C_{ha}$ stands for the
cost for each hour of the actor working; $N_h$ stands for the number of hours to
work.
\\[8pt]
It should be considered that, outside the academic scope, the costs for a
project inside a company are incremented due to social security taxes.
\section{Non-human costs}
The non-human costs of a project refer to all those that are not directly
associated with the individuals participating in its development but are equally
necessary for its successful completion. These costs encompass a wide range of
material resources and services required to carry out project tasks.
\\
In the case of the current software development project, non-human costs can be
classified into three main categories: hardware costs, infrastructure costs, and
indirect costs.
\subsection{Hardware}
The development of an application or a software design is required of a robust
and reliable infrastructure. This infrastructure is essential for the
implementation, testing and deployment of the application. The hardware costs
can vary in price and include an extensive variety of items for the development
and the testing of the product.
\\
In order to calculate the hardware costs, it has been taken into account the
indispensable for the project development. In terms of deployment
infrastructure, free-tier subscriptions have been used, meaning there is no
initial need for an investement in terms of infrastructure. For the hardware
elements, their life spand and their depreciation has also been taken into
account.
\\[12pt]
\begin{tabularx}{\textwidth}{|X|X|X|X|}
	\hline
	\rowcolor{rowColor}
	Hardware              & Life span              & Aprox. unit cost & Total cost     \\
	\hline
	Personal computer     & 10 years               & 2.000\texteuro   & 8.000\texteuro \\
	\hline
	Mouse                 & 5 years                & 20\texteuro      & 80\texteuro    \\
	\hline
	Keyboard              & 5 years                & 50\texteuro      & 200\texteuro   \\
	\hline
	Headset               & 5 years                & 40\texteuro      & 160\texteuro   \\
	\hline
	\multicolumn{3}{X|}{} & \textbf{8440\texteuro}                                     \\
	\cline{4-3}
\end{tabularx}
\captionof{table}{Hardware costs}
\vspace*{8pt}
Given this information, the total cost of each actor can be calculated through:
\begin{center}
	\[ T_c = C_u(\texteuro/h) * N_a \]
\end{center}
In this equation, $T_c$ stands for the total cost; $C_u$ stands for the cost for
each unit; $N_a$ stands for the number of actors to work.
\subsection{Infrastructure costs}
In term of infrastructure, the costs correpond to the cloud services offered by:
Google Cloud for the backend applications, and Vercel for the frontend. This
tools offer a free-tier or a free-credit tier, for educational purposes. The
team has taken advantage of this tiers to deploy the application.
\\[8pt]
Google Cloud offers a \$300 in credits to be used for 90 days. This allows the
usage of an extensive catalog of services. In this case, the Google Kubernetes
Engine (GKE) is the only service required. This service consists of a Kubernetes
cluster self-managed.
\\[8pt]
Vercel, on the other hand, offers a free plan which is targeted to personal and
educational projects. It includes unlimited bandwidth for static sites, 100GB
for dynamic sites, as well as offering unlimited deployments and serverless
functions.
\\[12pt]
\begin{tabularx}{\textwidth}{|X|X|X|X|}
	\hline
	\rowcolor{rowColor}
	Service               & Duration            & Pricing & Total cost \\
	\hline
	Google cloud          & 90 days             & \$300   & 0\texteuro \\
	\hline
	Vercel                & Unlimited           & \$0     & 0\texteuro \\
	\hline
	\multicolumn{3}{X|}{} & \textbf{0\texteuro}                        \\
	\cline{4-3}
\end{tabularx}
\captionof{table}{Infrastructure costs}
\subsection{Software expenses}
The development of a product does not only imply associated costs to the
infrastructure or hardware, but also to the tools and applications that allow
the developers and teams to do their job. This can include operative systems,
IDEs, project management software, databases, software libraries, frameworks,
and so on.
\\
It is important to note that such costs can vary a lot depending in the specific
needs of the project, and the desicions of the development team. In the case of
the project, given its academic focus, the development team has taken advantage
of free tools in order to minimize the costs.
\\[12pt]
\begin{tabularx}{\textwidth}{|X|X|}
	\hline
	\rowcolor{rowColor}
	Software                        & Cost                \\
	\hline
	Operative System                & 0\texteuro          \\
	\hline
	GitHub Organization             & 0\texteuro          \\
	\hline
	Discord                         & 0\texteuro          \\
	\hline
	IntellIJ Idea Community Edition & 0\texteuro          \\
	\hline
	Visual Studio Code              & 0\texteuro          \\
	\hline
	NeoVim                          & 0\texteuro          \\
	\hline
	Docker hub                      & 0\texteuro          \\
	\hline
	\multicolumn{1}{X|}{}           & \textbf{0\texteuro} \\
	\cline{2-2}
\end{tabularx}
\captionof{table}{Software costs}
\subsection{Indirect expenses}
In the execution of the project, it is not relevant to take into account costs
directly linked to the production of the final project, yet also include the
costs that are indirect to the development of the project: the indirect costs.
\\[8pt]
The indirect costs are crucial in the economic evaluation of a project,
otherwise, their omission can be translated in an underestimation of the
required resources.
\subsubsection{Electricity}
The electricity consumption encompasses the use of computers, electronic
devices, and workspace lighting. For this project, an average consumption of
0.05 Wh per laptop and 0.01 Wh for workspace lighting has been estimated, with a
total dedication time of 275 hours per team member for the master's thesis work.
\begin{center}
	\[ Consumption = Power * Time \]
\end{center}
\subsubsection{Internet connection}
The development team requires of constant internet connection to investigate,
communicate, access online resources and to develop. For this project, an
average cost of 25\texteuro has been considered for each member of the
team\cite{internet-cost}.
\subsubsection{Workspace rental}
The workspace is defined as the environment where each team member has carried
out their part of the project. This cost is calculated taking into account the
average rental price in Barcelona in 2022, which is estimated at 475\texteuro
per month\cite{workspace-cost}
\subsubsection{Office supplies}
Office supplies necessary for the project, such as notebooks and pens, also
constitute an indirect cost. For this project, an individual cost of 10\texteuro
has been assumed for the acquisition of these materials.
\subsubsection{Training and additional education}
Although there have been no additional training costs for this project, it is
important to consider it in the estimation of indirect costs. If the team needs
to acquire additional knowledge for the project through courses, this would be
an indirect cost to consider. However, given the wide range of free online
training available and the education received during the master's program, a
cost of 0\texteuro has been assumed.
\subsubsection{Total indirect costs}
The following table combines all the previous concepts:
\\[12pt]
\begin{tabularx}{\textwidth}{|X|X|X|}
	\hline
	\rowcolor{rowColor}
	Indirect cost         & Unit cost           & Total cost \\
	\hline
	Electricity usage     & X                   & 0\texteuro \\
	\hline
	Internet connection   & X                   & 0\texteuro \\
	\hline
	Workspace rent        & X                   & 0\texteuro \\
	\hline
	Office material       & X                   & 0\texteuro \\
	\hline
	Additional training   & 0\texteuro          & 0\texteuro \\
	\hline
	Travel expenses       & X                   & 0\texteuro \\
	\hline
	\multicolumn{2}{X|}{} & \textbf{0\texteuro}              \\
	\cline{3-3}
\end{tabularx}
\section{Contingencies}
Contingency costs in a project represent a provision of funds included in the
initial budget to address unforeseen events that were not anticipated during
project planning. These unforeseen events can arise from various situations,
such as technical difficulties discovered midway through the project,
realization of risks, changes in scope requested by the project sponsor,
initially unconsidered hidden requirements, among others. Therefore, it is
essential for the success of a project to have a margin of maneuver.
\\
In the case of our master's project, we are working in the field of informatics.
Based on the nature of this sector and previous experience, we have decided to
allocate 10\% of the total project budget to contingency costs. This percentage
reflects a balance between the risk we are willing to assume and the need to
keep the budget within acceptable limits.
\\[12pt]
\begin{tabularx}{\textwidth}{|X|X|X|}
	\hline
	\rowcolor{rowColor}
	Cost              & Contingency & Contingency cost \\
	\hline
	Electricity usage & X           & 0\texteuro       \\
	\hline
\end{tabularx}
\end{document}
