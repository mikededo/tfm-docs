\documentclass[../memory.tex]{subfiles}

\begin{document}
\chapter{Conclusions}
The flight information requirements for individuals who undertake frequent trips
are increasing as technology and access to data advance. Therefore, relying on
one-off search tools that require users to conduct individual searches to meet
their travel needs is no longer feasible, especially when they have fixed routes
that repeat in specific daily cycles, whether for work, academic, or established
project purposes.
\\[8pt]
The project and application have been developed using well-known patterns and
architectures widely used in companies and open-source projects. These
architectures have proven to meet the application requirements, providing
scalability, maintainability, and robustness. The hexagonal architecture
facilitated seamless integration with external services while maintaining a
layered and separated architecture. Additionally, the adoption of DDD
(domain-driven design) enabled consistent communication among the frontend,
backends, and product owner, ensuring a shared understanding of the system's
domain.
\\
The utilization of these patterns, ideas, and architectures has significantly
streamlined development time for both frontend and backend components, while
ensuring the application's maintainability and scalability.
\\[8pt]
In modern times, there are numerous solutions that simplify the deployment of
projects, ranging from simple to complex ones. These solutions often rely on
cloud services, such as Google Cloud or AWS, to name the most famous ones. In
this case, Google Cloud and Vercel have been chosen, simplifying the deployment
process and eliminating the need for local or internal servers. The utilization
of cloud services offers various benefits, including automated scalability, high
availability, efficient management of infrastructure resources, and built-in
security measures.
\\[8pt]
Alongside the cloud environment, all projects have taken advantage of
CI/CD (continuous integration/continuous development). The automation of code
building, testing and deployment have enabled faster delivery of new features
and bug fixes. Furthermore, it has fostered the team in terms of collaboration and
efficiency by ensuring stability and code quality \emph{throughout all stages of
	the development cycle}.
\\[8pt]
In the context of frontend development, the benefits of utilizing a hexagonal
architecture have become evident. Although it is not common to employ such an
architecture in frontend projects that use component-based frameworks, it has
demonstrated significant potential in various aspects. These advantages include
code sharing between frontends, simplified code splitting to reduce the amount
of JavaScript served to users, and improved maintainability and scalability of
domain-specific code.
\\
One particular note is the ability to completely separate domain logic and
implementation from the frontend application. This separation has streamlined
the frontend development experience and simplified UI testing, which is often
complex. However, it is important to acknowledge that companies aiming for rapid
product delivery may choose to keep business logic tightly coupled with the
frontend, as decoupling can be time-consuming.
\\[8pt]
Furthermore, the implementation of CI/CD in frontend development has reaffirmed
the importance of this practice in delivering higher code quality and a more
robust application.
\end{document}
