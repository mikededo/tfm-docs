\documentclass[a4paper, 11pt, oneside]{book}

\usepackage[english]{babel}
\usepackage[utf8]{inputenc}
\usepackage[a4paper, margin = 2.5cm, headheight=14pt]{geometry}
\usepackage{nopageno}
\usepackage{fontspec}

\date{}

\setmainfont{Helvetica Neue}[BoldFont = Helvetica Neue Bold]
\setmonofont[Scale=0.9]{JetBrains Mono}

\begin{document}
\section*{Abstract}
This master thesis presents the development of a Minimum Viable Product (MVP)
web application aimed at assisting frequent travelers in finding the best offers
for their recurrent routes. The project encompasses advanced software
architecture and design concepts, including Domain-Driven Design (DDD),
Hexagonal Architecture, and SOLID patterns. While the project is a group effort,
it particularly focuses on the frontend development while also addressing the
complexities of the backend.
\\[8pt]
Utilizing an iterative and collaborative development approach, the study
successfully implements the proposed architectural and design concepts to create
a robust and user-friendly MVP web application. The application incorporates a
comprehensive backend system to facilitate efficient search and retrieval of
cost-effective travel options tailored to users' recurrent routes.
\\[8pt]
The outcome of this thesis is a functional MVP web application, which showcases
the practical application of advanced software architecture and design
principles. The project highlights the challenges encountered during frontend
development and effectively addresses them to deliver a high-quality user
experience.
\\[16pt]
Keywords: web application development, frequent travelers, software
architecture, Domain-Driven Design (DDD), Hexagonal Architecture, SOLID design
principles, MVP (Minimum Viable Product), frontend development, backend
complexity, iterative development, user experience.
\newpage
\section*{Resum}
Aquesta tesi de màster presenta el desenvolupament d'una aplicació web de
Producte Viable Mínim (PVM) amb l'objectiu d'ajudar els viatgers freqüents a
trobar les millors ofertes per a les seves rutes recurrents. El projecte engloba
conceptes avançats d'arquitectura de programari i disseny, incloent
Domain-Driven Design (DDD), Arquitectura Hexagonal i patrons SOLID. Tot i que el
projecte és un treball en grup, es centra especialment en el desenvolupament del
frontend i, alhora, aborda les complexitats del backend.
\\[8pt]
Utilitzant un enfocament de desenvolupament iteratiu i col·laboratiu, l'estudi
implementa amb èxit els conceptes d'arquitectura i disseny proposats per crear
una aplicació web PVM robusta i fàcil d'utilitzar. L'aplicació incorpora un
sistema backend complet per facilitar la cerca eficient i la recuperació
d'opcions de viatge econòmiques adaptades a les rutes recurrents dels usuaris.
\\[8pt]
El resultat d'aquesta tesi és una aplicació web PVM funcional, que mostra
l'aplicació pràctica dels principis avançats d'arquitectura de programari i
disseny. El projecte posa de relleu els reptes trobats durant el desenvolupament
del frontend i els aborda amb eficàcia per proporcionar una experiència d'usuari
de qualitat.
\\[16pt]
Paraules clau: desenvolupament d'aplicacions web, viatgers freqüents,
arquitectura de programari, Domain-Driven Design (DDD), Arquitectura Hexagonal,
principis de disseny SOLID, PVM (Producte Viable Mínim), desenvolupament de
frontend, desenvolupament de backend, desenvolupament iteratiu, experiència
d'usuari.
\newpage
\section*{Resumen}
Esta tesis de máster presenta el desarrollo de una aplicación web de Producto
Mínimo Viable (PMV) con el objetivo de ayudar a los viajeros frecuentes a
encontrar las mejores ofertas para sus rutas recurrentes. El proyecto abarca
conceptos avanzados de arquitectura y diseño de software, incluyendo
Domain-Driven Design (DDD), Arquitectura Hexagonal y patrones SOLID. Si bien el
proyecto es un esfuerzo grupal, se centra especialmente en el desarrollo del
frontend y también aborda las complejidades del backend.

Utilizando un enfoque de desarrollo iterativo y colaborativo, el estudio
implementa con éxito los conceptos de arquitectura y diseño propuestos para
crear una aplicación web PMV robusta y fácil de usar. La aplicación incorpora un
sistema backend completo para facilitar la búsqueda eficiente y la recuperación
de opciones de viaje económicas adaptadas a las rutas recurrentes de los
usuarios.
\\[8pt]
El resultado de esta tesis es una aplicación web PMV funcional, que muestra la
aplicación práctica de los principios avanzados de arquitectura y diseño de
software. El proyecto destaca los desafíos encontrados durante el desarrollo del
frontend y los aborda de manera efectiva para ofrecer una experiencia de usuario
de alta calidad.
\\[16pt]
Palabras clave: desarrollo de aplicaciones web, viajeros frecuentes,
arquitectura de software, Domain-Driven Design (DDD), Arquitectura Hexagonal,
principios de diseño SOLID, PMV (Producto Mínimo Viable), desarrollo de
frontend, desarrollo de backend, desarrollo iterativo, experiencia de usuario.
\end{document}
